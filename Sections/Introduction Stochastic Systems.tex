So far we have dealt with systems that are purely \emph{deterministic}, which means that for two identical systems with \emph{exactly} the same initial conditions, their trajectories will be the same for all times.
However, for the case of deterministic chaos, a tiny difference in the initial condition can lead to a completely different \emph{long-term behavior}. In contrast to deterministic systems, for stochastic systems not even the \emph{short-term behavior} is predictable, not even in principle, because there are forces at work that are outside of our control.\\
All relevant systems in nature contain both deterministic and stochastic elements, and the question is simply which part, if any, is \emph{dominating}.\\
If there were no random forces, effects like the \emph{spontaneous breaking of symmetry} discussed in section (\ref{sec:spcpbf}) could not occur.
If a potential landscape switches from a single minimum to a double-well as in Figure (\ref{fig:spbf}), nothing would happen, the system would simply sit at the unstable fixed point until someone comes along and kicks it a little.\\
The \textbf{stochastic force} allows the system to explore all of its phase space as we shall see.
Without such a force the system is restricted to a single trajectory from an initial condition to an attractor or to infinity, or is even left stranded at an unstable equilibrium point.