Time-discrete maps are useful to study in order to understand the complexity that can originate from simple iterative procedures.
Maps are much easier to implement numerically than differential equations and exhibit an incredible richness of complex structures.
These systems are known variously as difference equations, recursion relations, iterated maps, or simply maps.
In general, a discrete map is given by
\begin{equation}
	\mathbf{x}_{n+1}=\mathbf{f}(\mathbf{x}_n)
\end{equation}
where $\mathbf{x}$ and $\mathbf{f}$ are $m-$dimensional vectors.\\
The sequence $x_0,x_1,x_2,\ldots$ is called the {\textbf{orbit}} starting from $x_0$.\\\\
\textbf{Fixed points} occur when $\mathbf{x}_{n+1}=\mathbf{f}(\mathbf{x}_n)=\mathbf{x}_n$ for all $n$.
\subsection{Recurrence Relations}
\subsubsection{First-Order Difference Equations}
A \emph{recurrence relation} can be defined by a difference equation of the form
\begin{equation}{\label{eq:fodels}}
	x_{n+1}=f(x_n)
\end{equation}
where $x_{n+1}$ is derived from $x_n$ and $n=0, 1, 2, 3,\ldots$\\
If one starts with an initial value, say, $x_0$, then iteration of Equation (\ref{eq:fodels}) leads to a sequence of the form
\begin{equation}
	\{x_i:i=0\ to\ \infty\}=\{x_0,x_1,x_2,\ldots x_n,x_{n+1}\ldots\}
\end{equation}
\begin{theorem}
	The general solution of the first-order linear difference equation
	\begin{equation}
		x_{n+1}=mx_n+c,\ n=0,1,2,3,\ldots,
	\end{equation}
	is given by
	\begin{equation}
		x_n=m^nx_0+
		\begin{cases}
			\dfrac{m^n-1}{m-1}c& \text{if}\ m\neq1\\
			nc& \text{if}\ m=1
		\end{cases}
	\end{equation}
\end{theorem}
\subsubsection{Second-Order Linear Difference Equations}
Recurrence relations involving terms whose suffices differ by two are known as \emph{second-order linear difference equations}.
The general form of these equations with constant coefficients is
\begin{equation}{\label{eq:sodels}}
	ax_{n+2}=bx_{n+1}+cx_n
\end{equation}
\begin{theorem}
	The general solution of the second-order recurrence relation (\ref{eq:sodels}) is
	\begin{equation}
		x_n=
		\begin{cases}
			k_1\lambda_1^n+k_2\lambda_2^n& \text{if}\ \lambda_1\neq\lambda_2\\
			(k_3+nk_4)\lambda^n& \text{if}\ \lambda_1=\lambda_2=\lambda\ (\text{say})
		\end{cases}\quad\text{where}\quad a\lambda^2-b\lambda-c=0
	\end{equation}
	Note that when $\lambda_1\ and\ \lambda_2$ are complex, the general solution can be expressed as
	\begin{equation}
		x_n=k_1\lambda_1^n+k_2\lambda_2^n=k_1(re^{i\theta})^n+k_2(re^{-i\theta})^n=r^n(A\cos(n\theta)+B\sin(n\theta))
	\end{equation}
	where $A$ and $B$ are constants.
	When the roots are complex, the solution oscillates and is real.
\end{theorem}